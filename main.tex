\documentclass[a4paper,12pt,twoside]{scrreprt}
% Autor der Vorlage: Klaus Rheinberger, FH Vorarlberg, 2017-02-20

% Pakete:
\usepackage[utf8]{inputenc}
\usepackage[T1]{fontenc} % Silbentrennung bei Sonderzeichen
\usepackage{graphicx} % Bilder einbinden
\usepackage{wrapfig} % Bilder positionieren
\usepackage[ngerman]{babel} % Deutsche Sprachanpassungen
\usepackage{minted} % Code Highlighting/Import
\usepackage{csquotes} % Anführungszeichen und Zitieren
\usepackage[bindingoffset=8mm]{geometry} % Bindeverlust von 8mm einbeziehen
\usepackage{caption} % Abbildungslegenden
\usepackage{xcolor} % Farbige Hervorhebungen
\usepackage{setspace} % Zeilenabstand
\usepackage[style=authoryear,citestyle=authoryear,backend=biber]{biblatex} % Literaturverweise
\usepackage[
    linktocpage=true,
    pdfauthor={Dominic Luidold},
    pdftitle={TODO}
]{hyperref} % Links -> \href{https://www.wikibooks.org}{Wikibooks home}
\usepackage{acronym} % Abkürzungsverzeichnis

% Einstellungen:
\captionsetup{format=hang, justification=raggedright}
\addbibresource{Zotero.bib}
\setcounter{secnumdepth}{4}
\setcounter{tocdepth}{4} % Tiefe der Gliederung im Inhaltsverzeichnis

% Custom Commands
\renewcommand{\listingscaption}{Quellcode}
\renewcommand\listoflistingscaption{Quellcodeverzeichnis}

% Dokumentenbeginn
\begin{document}
\onehalfspacing % Zeilenabstand 1,5

% Titelblatt:
% \newpage\mbox{}\newpage
\cleardoublepage % force output to a right page
\thispagestyle{empty}
\begin{titlepage}
    \begin{flushright}
    \includegraphics[width=0.4\linewidth]{images/Logo_FHV.jpg}
    \end{flushright}
    \begin{flushleft}
    \section*{TBD}
    \vspace{1cm}

    Bachelorarbeit II\\
    zur Erlangung des akademischen Grades
    \vspace{0.5cm}

    \textbf{Bachelor of Science in Engineering (BSc)}

    \vspace{1cm}
    Fachhochschule Vorarlberg\newline
    Informatik – Software and Information Engineering

    \vspace{0.5cm}

    Betreut von\newline
    Prof. (FH) Dipl. Inform. Thomas Feilhauer

    \vspace{0.5cm}

    Vorgelegt von\newline
    Dominic Luidold\newline
    Dornbirn, 20. Mai 2021
    \end{flushleft}
\end{titlepage}

% Widmung:
\newpage
\section*{Widmung}
\label{sec:widmung}
TODO

\bigskip

\begin{quote}
    \begin{flushright}
        \textit{\enquote{TODO}}\\
        TODO
    \end{flushright}
\end{quote}

% Kurzreferat:
\newpage
\section*{Kurzreferat}
\label{sec:kurzreferat}

\subsection*{TODO}

TODO

% Abstract:
\newpage
\section*{Abstract}
\label{sec:abstract}

\subsection*{TODO}

TODO

% Geschlechtergerechte Sprache:
\newpage
\section*{Geschlechtergerechte Sprache}
\label{sec:gendern}

Der Verfasser der vorliegenden Arbeit bekennt sich zu einer geschlechtergerechten Sprachverwendung.

Um die Lesbarkeit zu gewährleisten und zugunsten der Textökonomie werden die verwendeten Personen beziehungsweise Personengruppen fix männlich oder weiblich zugeordnet. Zum Beispiel wird immer \enquote{die Entwicklerin} und \enquote{der Benutzer} verwendet. Es wurde besonders darauf geachtet, stereotype Rollenbeschreibungen zu vermeiden. Die insgesamt eventuell dadurch hervorgerufene Irritation bei den Lesenden ist gewünscht und soll dazu beitragen, eine Bewusstheit für die bestehende, Frauen diskriminierende Sprachgewohnheit (generelle Verwendung der männlichen Begriffe für beide Geschlechter) zu wecken beziehungsweise zu stärken.

% Inhaltsverzeichnis:
\cleardoublepage % force output to a right page
\setcounter{tocdepth}{2}
\tableofcontents

\clearpage
\phantomsection
\addcontentsline{toc}{chapter}{Abbildungsverzeichnis}
\listoffigures

% Abkürzungsverzeichnis:
\clearpage
\phantomsection
\addcontentsline{toc}{chapter}{Abkürzungsverzeichnis}
\chapter*{Abkürzungsverzeichnis}
\begin{acronym}
  \acro{MPA}{Multi-page Application}
  \acro{SPA}{Single-page Application}
\end{acronym}

\chapter{Einleitung}
\label{chap:einleitung}
Diese Bachelorarbeit verfolgt das Ziel, einen Einblick in die \ac{SPA} Frameworks \textit{Angular}\footnote{\href{https://angular.io/}{Angular (https://angular.io)}} und \textit{Vaadin}\footnote{\href{https://vaadin.com/}{Vaadin (https://vaadin.com)}} zu geben und deren Gemeinsamkeiten, Unterschiede sowie Vor- und Nachteile zu beleuchten.

\medskip

Um ein grundlegendes Verständnis über die Thematik von \aclp{SPA} zu erlangen, wird zu Beginn der Arbeit auf das Konzept einer \ac{SPA} eingegangen und die zugrundeliegende Herangehensweise mit der einer klassischen \ac{MPA} beziehungsweise eines Rich Clients verglichen. Im weiteren Verlauf werden die unterschiedlichen Ansätze von Angular und Vaadin genauer betrachtet und eine tatsächliche Umsetzung der zuvor erläuterten Technologien mittels zweier Demo-Applikationen getestet. Am Ende dieser Arbeit wir darauf eingegangen, ob sich - anhand unterschiedlicher Kriterien und Anwendungsfälle - eine Empfehlung für eines der beiden \acs{SPA} Frameworks aussprechen lässt.

\section{Motivation}
\label{sec:motivation}
In den letzten Jahren lässt sich beobachten, dass Webapplikationen, Apps und Anwendungen allgemein verstärkt mittels des \acs{SPA}-Ansatzes umgesetzt werden und somit auf einen Thin Client - im Gegensatz zu klassischeren Rich Clients und \aclp{MPA} - setzen. Für die Umsetzung einer solchen Applikation stehen eine Vielzahl von Frameworks zur Verfügung, die darüber hinaus weitere Features bieten und Entwickler:innen bei der Umsetzung unterstützen.

\newpage

Die richtige Wahl des Frameworks, der jeweiligen Technologien und der im Hintergrund agierenden Strukturen spielen eine wesentliche Rolle bei der Planung und Umsetzung eines neuen Projektes. Welches Framework sich besser eignet, lässt sich oftmals nicht auf den ersten (oder sogar zweiten) Blick feststellen. Diese Arbeit befasst sich daher genauer mit dem Konzept von \aclp{SPA} und vergleicht zwei darauf aufbauende Frameworks, die mit deutlich unterschiedlichen Technologie-Stacks arbeiten und zu vergleichbaren Lösungen führen.

\section{Problemstellung}
\label{sec:problemstellung}
Die in Abschnitt \ref{sec:motivation} auf Seite \pageref{sec:motivation} angesprochene Vielzahl an \acs{SPA}-Frameworks bietet grundlegend den Vorteil, dass eine große Auswahlmöglichkeit und eine gewisse Konkurrenz untereinander zu einem hohen Qualitätsstandard führt. Zudem wird dadurch sichergestellt, dass es für jedes Projekt - unabhängig von den jeweiligen Anforderungen und etwaigen Eigenheiten - eine Möglichkeit gibt, dieses mit einem der verfügbaren Frameworks umzusetzen. Auf der anderen Seite führt die stetig wachsende Anzahl an Möglichkeiten jedoch dazu, dass sich meist nur schwer beurteilen lässt, welches Framework sich für die Umsetzung einer Applikation bestmöglich eignet.

\begin{figure}[ht]
    \centering
    \includegraphics[scale=0.5]{images/js-frameworks.png}
    \caption[Liste möglicher JavaScript-Frameworks zur Umsetzung von \aclp{SPA}]{Liste möglicher JavaScript-Frameworks zur Umsetzung von \aclp{SPA}. Quelle: \cite{a_best_2020}}
    \label{fig:js-frameworks}
\end{figure}

Neben 

% Literaturverzeichnis:
\clearpage
\phantomsection
\addcontentsline{toc}{chapter}{Literaturverzeichnis}
\printbibliography

\chapter*{Eidesstattliche Erklärung}
\addcontentsline{toc}{chapter}{Eidesstattliche Erklärung}
Ich erkläre hiermit an Eides statt, dass ich die vorliegende Bachelorarbeit I selbstständig und ohne Benutzung anderer als der angegebenen Hilfsmittel angefertigt habe. Die aus fremden Quellen direkt oder indirekt übernommenen Stellen sind als solche kenntlich gemacht. Die Arbeit wurde bisher weder in gleicher noch in ähnlicher Form einer anderen Prüfungsbehörde vorgelegt und auch noch nicht veröffentlicht.

\vspace{5cm}
\noindent
Dornbirn, am 20. Mai 2021\hfill Dominic Luidold

\end{document}
