\documentclass[a4paper,12pt,twoside]{scrreprt}
% Autor der Vorlage: Klaus Rheinberger, FH Vorarlberg, 2017-02-20

% Pakete:
\usepackage[utf8]{inputenc}
\usepackage[T1]{fontenc} % Silbentrennung bei Sonderzeichen
\usepackage{graphicx} % Bilder einbinden
\usepackage{wrapfig} % Bilder positionieren
\usepackage[ngerman]{babel} % Deutsche Sprachanpassungen
\usepackage{minted} % Code Highlighting/Import
\usepackage{csquotes} % Anführungszeichen und Zitieren
\usepackage[bindingoffset=8mm]{geometry} % Bindeverlust von 8mm einbeziehen
\usepackage{caption} % Abbildungslegenden
\usepackage{xcolor} % Farbige Hervorhebungen
\usepackage{setspace} % Zeilenabstand
\usepackage[style=authoryear,citestyle=authoryear,backend=biber]{biblatex} % Literaturverweise
\usepackage[
    linktocpage=true,
    pdfauthor={Dominic Luidold},
    pdftitle={TODO}
]{hyperref} % Links -> \href{https://www.wikibooks.org}{Wikibooks home}
\usepackage{acronym} % Abkürzungsverzeichnis

% Einstellungen:
\captionsetup{format=hang, justification=raggedright}
\addbibresource{Zotero.bib}
\setcounter{secnumdepth}{4}
\setcounter{tocdepth}{4} % Tiefe der Gliederung im Inhaltsverzeichnis

% Custom Commands
\renewcommand{\listingscaption}{Quellcode}
\renewcommand\listoflistingscaption{Quellcodeverzeichnis}

% Dokumentenbeginn
\begin{document}
\onehalfspacing % Zeilenabstand 1,5

% Titelblatt:
% \newpage\mbox{}\newpage
\cleardoublepage % force output to a right page
\thispagestyle{empty}
\begin{titlepage}
    \begin{flushright}
    \includegraphics[width=0.4\linewidth]{images/Logo_FHV.jpg}
    \end{flushright}
    \begin{flushleft}
    \section*{TBD}
    \vspace{1cm}

    Bachelorarbeit II\\
    zur Erlangung des akademischen Grades
    \vspace{0.5cm}

    \textbf{Bachelor of Science in Engineering (BSc)}

    \vspace{1cm}
    Fachhochschule Vorarlberg\newline
    Informatik – Software and Information Engineering

    \vspace{0.5cm}

    Betreut von\newline
    Prof. (FH) Dipl. Inform. Thomas Feilhauer

    \vspace{0.5cm}

    Vorgelegt von\newline
    Dominic Luidold\newline
    Dornbirn, \colorbox{yellow}{20. Mai 2021}
    \end{flushleft}
\end{titlepage}

% Widmung:
\newpage
\section*{Widmung}
\label{sec:widmung}
TODO

\bigskip

\begin{quote}
    \begin{flushright}
        \textit{\enquote{TODO}}\\
        TODO
    \end{flushright}
\end{quote}

% Kurzreferat:
\newpage
\section*{Kurzreferat}
\label{sec:kurzreferat}

\subsection*{TODO}

TODO

% Abstract:
\newpage
\section*{Abstract}
\label{sec:abstract}

\subsection*{TODO}

TODO

% Geschlechtergerechte Sprache:
\newpage
\section*{Geschlechtergerechte Sprache}
\label{sec:gendern}

Der Verfasser der vorliegenden Arbeit bekennt sich zu einer geschlechtergerechten Sprachverwendung.

Um die Lesbarkeit zu gewährleisten und zugunsten der Textökonomie werden die verwendeten Personen beziehungsweise Personengruppen fix männlich oder weiblich zugeordnet. Zum Beispiel wird immer \enquote{die Entwicklerin} und \enquote{der Benutzer} verwendet. Es wurde besonders darauf geachtet, stereotype Rollenbeschreibungen zu vermeiden. Die insgesamt eventuell dadurch hervorgerufene Irritation bei den Lesenden ist gewünscht und soll dazu beitragen, eine Bewusstheit für die bestehende, Frauen diskriminierende Sprachgewohnheit (generelle Verwendung der männlichen Begriffe für beide Geschlechter) zu wecken beziehungsweise zu stärken.

% Inhaltsverzeichnis:
\cleardoublepage % force output to a right page
\setcounter{tocdepth}{2}
\tableofcontents

\clearpage
\phantomsection
\addcontentsline{toc}{chapter}{Abbildungsverzeichnis}
\listoffigures

% Abkürzungsverzeichnis:
\clearpage
\phantomsection
\addcontentsline{toc}{chapter}{Abkürzungsverzeichnis}
\chapter*{Abkürzungsverzeichnis}
\begin{acronym}
  \acro{MPA}{Multi-page Application}
  \acro{PWA}{Progressive Web App}
  \acro{SPA}{Single-page Application}
  \acro{UI}{User Interface}
\end{acronym}

\chapter{Einleitung}
\label{chap:einleitung}
Diese Bachelorarbeit verfolgt das Ziel, einen Einblick in die \ac{SPA} Frameworks \textit{Angular}\footnote{\href{https://angular.io/}{Angular (https://angular.io)}} und \textit{Vaadin}\footnote{\href{https://vaadin.com/}{Vaadin (https://vaadin.com)}} zu geben und deren Gemeinsamkeiten, Unterschiede sowie Vor- und Nachteile zu beleuchten.

\medskip

Um ein grundlegendes Verständnis über die Thematik von \aclp{SPA} zu erlangen, wird zu Beginn der Arbeit auf das Konzept einer \ac{SPA} eingegangen und die zugrundeliegende Herangehensweise mit der einer klassischen \ac{MPA} verglichen. Im weiteren Verlauf werden die unterschiedlichen Ansätze von Angular und Vaadin genauer betrachtet und eine tatsächliche Umsetzung der zuvor erläuterten Technologien mittels zweier Demo-Applikationen getestet. Am Ende dieser Arbeit wir darauf eingegangen, ob sich - anhand unterschiedlicher Kriterien und Anwendungsfälle - eine Empfehlung für eines der beiden \acs{SPA} Frameworks aussprechen lässt.

\section{Motivation}
\label{sec:motivation}
In den letzten Jahren lässt sich beobachten, dass Webapplikationen, Apps und Anwendungen allgemein verstärkt mittels des \acs{SPA}-Ansatzes umgesetzt werden und somit auf einen Thin Client - im Gegensatz zu klassischeren \aclp{MPA} - setzen. Für die Umsetzung einer solchen Applikation stehen eine Vielzahl von Frameworks zur Verfügung, die darüber hinaus weitere Features bieten und \colorbox{yellow}{Entwickler:innen} bei der Umsetzung unterstützen.

\newpage

Die richtige Wahl des Frameworks, der jeweiligen Technologien und der im Hintergrund agierenden Strukturen spielen eine wesentliche Rolle bei der Planung und Umsetzung eines neuen Projektes. Welches Framework sich besser eignet, lässt sich oftmals nicht auf den ersten (oder sogar zweiten) Blick feststellen. Diese Arbeit befasst sich daher genauer mit dem Konzept von \aclp{SPA} und vergleicht zwei darauf aufbauende Frameworks, die mit deutlich unterschiedlichen Technologie-Stacks arbeiten und zu vergleichbaren Lösungen führen.

\section{Problemstellung}
\label{sec:problemstellung}
Die in Abschnitt \ref{sec:motivation} auf Seite \pageref{sec:motivation} angesprochene Vielzahl an \acs{SPA}-Frameworks bietet grundlegend den Vorteil, dass eine große Auswahlmöglichkeit und eine gewisse Konkurrenz untereinander zu einem hohen Qualitätsstandard führt. Zudem wird dadurch sichergestellt, dass es für jedes Projekt - unabhängig von den jeweiligen Anforderungen und etwaigen Eigenheiten - eine Möglichkeit gibt, dieses mit einem der verfügbaren Frameworks umzusetzen. Auf der anderen Seite führt die stetig wachsende Anzahl an Möglichkeiten jedoch dazu, dass sich meist nur schwer beurteilen lässt, welches Framework sich für die Umsetzung einer Applikation bestmöglich eignet.

\begin{figure}[ht]
    \centering
    \includegraphics[scale=0.5]{images/js-frameworks.png}
    \caption[Liste möglicher JavaScript-Frameworks zur Umsetzung von \aclp{SPA}]{Liste möglicher JavaScript-Frameworks zur Umsetzung von \aclp{SPA} (Quelle: \cite{a_best_2020})}
    \label{fig:js-frameworks}
\end{figure}

Um eine geeignete Wahl eines Frameworks treffen zu können, sollten vorab Kriterien und Anforderungen definiert werden, die schlussendlich erfüllt werden müssen. Neben den projektspezifischen Eigenheiten, die in den meisten Fällen von einer Vielzahl der Frameworks abgedeckt werden können, stellt sich die Auswahl der zugrundeliegenden Technologien als eine der wichtigsten Herausforderungen dar. Diese Entscheidung muss gut überlegt und abgewogen werden, da diese im weiteren Verlauf weitreichende Folgen bei der Umsetzung einer (Web-) Applikation zur Folge hat und sich ein Wechsel nach gestarteter Entwicklung nur unter großem Aufwand umsetzen lässt.

Die in Abbildung \ref{fig:js-frameworks} auf Seite \pageref{fig:js-frameworks} dargestellten Frameworks zeigen eine Auswahl an Frameworks auf, die auf \textit{JavaScript} aufbauen beziehungsweise basieren und somit primär auf dem Client - dem Browser -  eingesetzt werden können. \aclp{SPA} lassen sich jedoch nicht nur Frontend-seitig entwickeln (bei denen ein Großteil der Logik auf einem externen Server abläuft), sondern können ebenfalls mittels auf \textit{Java} basierenden Frameworks umgesetzt werden. Bei diesen Frameworks - zu denen unter anderem Vaadin gehört - lässt sich sowohl die Logik als auch das \ac{UI}, stellenweise gänzlich, kombinieren.

\medskip

Da die unterschiedlichen Ansätze, sowohl hinter Vaadin als auch Angular, gewisse Vor- und Nachteile sowie Tücken mit sich bringen, fällt die Wahl auf eines der beiden \acs{SPA}-fähigen Frameworks auf den ersten Blick nicht leicht. Hinzu kommt die Frage, welches der Frameworks weiterführende Funktionalitäten bietet, um mit geringem Aufwand beispielsweise eine \ac{PWA} umzusetzen oder anwendungsspezifische Daten lokal sowie extern persistieren zu können.

\section{Zielsetzung}
\label{sec:zielsetzung}
Die in den Abschnitten \ref{sec:motivation} und \ref{sec:problemstellung} angeführten Punkte haben aufgezeigt, dass die große Anzahl an Frameworks, mit denen \aclp{SPA} umgesetzt werden können, zwar sehr positiv einzuschätzen ist, die damit verbundenen Probleme bei der Auswahl des richtigen Frameworks werden dadurch jedoch verstärkt. Aufgrund der unterschiedlichen zugrundeliegenden Technologien und einhergehenden Herangehensweisen ist eine bedachte Wahl wichtig.

\newpage

Diese Arbeit verfolgt daher das Ziel, das JavaScript Framework \textit{Angular} dem auf Java basierenden Framework \textit{Vaadin} gegenüberzustellen und zu vergleichen. Das Ziel ist es, mittels Literatur belegter Vergleiche einen Allgemeinen Überblick über \aclp{SPA} zu geben, diese klassischen Ansätzen gegenüberzustellen und zwei Demo-Applikationen zu entwickeln. Diese Webanwendungen werden dann herangezogen, um anhand von vorab definierten Kriterien feststellen zu können, ob und in wie weit Empfehlungen für eines der beiden Frameworks ausgesprochen werden kann.

\medskip

Um den Fokus dieser Arbeit genauer zu definieren und einzuschränken, wird die Planung, Umsetzung sowie abschließenden Beurteilung der Applikationen anhand der ausgearbeiteten Kriterien auf folgende Punkte beschränkt:
\begin{itemize}
    \item Möglichkeit zur einfachen Umsetzung einer \acf{PWA}
    \item Möglichkeit der Wiederverwendbarkeit von Komponenten, gegebenenfalls mittels \textit{Web Components}
    \item Möglichkeit Daten lokal (Browser) sowie extern (Server) zu persistieren
\end{itemize}

\chapter{Stand der Technik}
\label{chap:stand-technik}
Das folgende Kapitel gibt einen Überblick über die Funktionsweise einer \acl{SPA} und vergleicht das Konzept von \acsp{SPA} mit klassischen \aclp{MPA}. Im Anschluss wird im Detail auf die Funktionsweise von Angular und Vaadin beziehungsweise deren unterschiedlichen Ansatz in Hinblick auf Client und Server eingegangen und die jeweiligen Vor- und Nachteile genauer beleuchtet.

\section{Konzept einer \acl{SPA}}
\label{sec:konzept-spa}


% Literaturverzeichnis:
\clearpage
\phantomsection
\addcontentsline{toc}{chapter}{Literaturverzeichnis}
\printbibliography

\chapter*{Eidesstattliche Erklärung}
\addcontentsline{toc}{chapter}{Eidesstattliche Erklärung}
Ich erkläre hiermit an Eides statt, dass ich die vorliegende Bachelorarbeit I selbstständig und ohne Benutzung anderer als der angegebenen Hilfsmittel angefertigt habe. Die aus fremden Quellen direkt oder indirekt übernommenen Stellen sind als solche kenntlich gemacht. Die Arbeit wurde bisher weder in gleicher noch in ähnlicher Form einer anderen Prüfungsbehörde vorgelegt und auch noch nicht veröffentlicht.

\vspace{5cm}
\noindent
Dornbirn, am \colorbox{yellow}{20. Mai 2021}\hfill Dominic Luidold

\end{document}
